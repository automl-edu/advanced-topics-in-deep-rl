\documentclass[12pt]{beamer}

\usepackage[utf8]{inputenc}
\usepackage{amsmath}
\usepackage{hyperref}
\usepackage{graphicx}
\usepackage{caption}
\usepackage{subcaption}
\usepackage[style=ext-authoryear, backend=biber]{biblatex}
\addbibresource{refs.bib}
\usepackage{csquotes}

\DeclareOuterCiteDelims{parencite}{\bibopenbracket}{\bibclosebracket}

\begin{document}

\begin{frame}{Proximal Curriculum Learning with State Novelty}
  \begin{itemize}
    \item Use zone of proximal development (pick start state with probability of success $PoS \approx 0.5$) \parencite{proximal}
    \item approximate $PoS$ via value-function \parencite{proximal}
      \pause
    \item \textit{Own Idea:} already using state to guide student, why not pick novel states
    \item[$\rightarrow$] should help value-function converge
    \item[$\rightarrow$] should still pick mostly promising states
      \pause
    \item trade off via hyperparameter between $PoS$ and novelty
    \item alternatively incorporate state novelty as intrinsic reward: directly influences $PoS$
  \end{itemize}
\end{frame}

\begin{frame}{Practical Considerations}
  \begin{itemize}
    \item \textit{Implementation Overhad:} no given code, but modifications (including hyperparameters) to stable-baselines listed for \parencite{proximal}
      \pause
    \item \textit{Compute Overhead:} should be reasonably cheap depending on the state novelty algorithm and environment
      \pause
    \item \textit{Reproducibility:} Should reproduce given limited modifications to stable-baselines. If not try to use rollouts to approximate $PoS$.
      \pause
    \item MiniGrid should be good, sparse and many states
  \end{itemize}
\end{frame}
\end{document}
